\subsubsection{Blockchain} \label{sec:blockchain}
The blockchain has to fulfill different requirements.
\begin{itemize}
	\item The blockchain mainly has to be installed on arch-linux computers. So the installation should be easily on this platform.
	\item The blockchain has to have a interface, which allows to use the blockchain in background cause the blockchain should be started with a systemd service.
	\item The blockchain has to support smart contracts for the functional requirements (mainly get/commit hashes).
	\item The blockchain should support private networks, cause the aursec-network should only contain transactions related to aursec.
	\item This private network should be easily build.
\end{itemize}

The blockchain of choice was a ethereum-blockchain. The main reason for this blockchain, are the solidity smart contracts. These contracts are easy to write and understand. 

\paragraph*{smart-contracts}
Aursec has two smart contracts. The first one is a formal contract, which allows the owner of the contract to delete the contract. The second contract is a child of the first. In this contract are the ``user-functions'', which allows all users to send hash-commits and request the current consensus hash of a versionized package. 

\paragraph*{network}
Currently the nodes connect to the network over a bootnode. This bootnode runs on a 24/7 server.

%TODO add more refs
\paragraph*{interfaces}
The Ethereum-blockchain uses 2 different interfaces. The IPC (interprocess communication) and the RPC (remote procedure call) interface. The IPC interface of the aursec-blockchain is deactivated by default, cause the user shouldn't have the need of a console which is attached to the running node. The user interact with the blockchain through shellscripts especially aursec-chain and the pythonscript aursec-tui [see \ref{sec:aursec-chain} ]. These scripts communicate with the blockchain trough the RPC interface. 